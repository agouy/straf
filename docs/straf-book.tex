% Options for packages loaded elsewhere
\PassOptionsToPackage{unicode}{hyperref}
\PassOptionsToPackage{hyphens}{url}
%
\documentclass[
]{book}
\usepackage{lmodern}
\usepackage{amssymb,amsmath}
\usepackage{ifxetex,ifluatex}
\ifnum 0\ifxetex 1\fi\ifluatex 1\fi=0 % if pdftex
  \usepackage[T1]{fontenc}
  \usepackage[utf8]{inputenc}
  \usepackage{textcomp} % provide euro and other symbols
\else % if luatex or xetex
  \usepackage{unicode-math}
  \defaultfontfeatures{Scale=MatchLowercase}
  \defaultfontfeatures[\rmfamily]{Ligatures=TeX,Scale=1}
\fi
% Use upquote if available, for straight quotes in verbatim environments
\IfFileExists{upquote.sty}{\usepackage{upquote}}{}
\IfFileExists{microtype.sty}{% use microtype if available
  \usepackage[]{microtype}
  \UseMicrotypeSet[protrusion]{basicmath} % disable protrusion for tt fonts
}{}
\makeatletter
\@ifundefined{KOMAClassName}{% if non-KOMA class
  \IfFileExists{parskip.sty}{%
    \usepackage{parskip}
  }{% else
    \setlength{\parindent}{0pt}
    \setlength{\parskip}{6pt plus 2pt minus 1pt}}
}{% if KOMA class
  \KOMAoptions{parskip=half}}
\makeatother
\usepackage{xcolor}
\IfFileExists{xurl.sty}{\usepackage{xurl}}{} % add URL line breaks if available
\IfFileExists{bookmark.sty}{\usepackage{bookmark}}{\usepackage{hyperref}}
\hypersetup{
  pdftitle={The STRAF Book},
  pdfauthor={Alexandre Gouy and Martin Zieger},
  hidelinks,
  pdfcreator={LaTeX via pandoc}}
\urlstyle{same} % disable monospaced font for URLs
\usepackage{longtable,booktabs}
% Correct order of tables after \paragraph or \subparagraph
\usepackage{etoolbox}
\makeatletter
\patchcmd\longtable{\par}{\if@noskipsec\mbox{}\fi\par}{}{}
\makeatother
% Allow footnotes in longtable head/foot
\IfFileExists{footnotehyper.sty}{\usepackage{footnotehyper}}{\usepackage{footnote}}
\makesavenoteenv{longtable}
\usepackage{graphicx,grffile}
\makeatletter
\def\maxwidth{\ifdim\Gin@nat@width>\linewidth\linewidth\else\Gin@nat@width\fi}
\def\maxheight{\ifdim\Gin@nat@height>\textheight\textheight\else\Gin@nat@height\fi}
\makeatother
% Scale images if necessary, so that they will not overflow the page
% margins by default, and it is still possible to overwrite the defaults
% using explicit options in \includegraphics[width, height, ...]{}
\setkeys{Gin}{width=\maxwidth,height=\maxheight,keepaspectratio}
% Set default figure placement to htbp
\makeatletter
\def\fps@figure{htbp}
\makeatother
\setlength{\emergencystretch}{3em} % prevent overfull lines
\providecommand{\tightlist}{%
  \setlength{\itemsep}{0pt}\setlength{\parskip}{0pt}}
\setcounter{secnumdepth}{5}
\usepackage[T1]{fontenc}
\usepackage[utf8]{inputenc}
\usepackage{geometry}
\geometry{a5paper}
\usepackage{fontspec}
\setmainfont[Ligatures=TeX,Scale=0.8]{Arial}

\usepackage{color}
\usepackage{framed}
\setlength{\fboxsep}{.8em}

\usepackage{tcolorbox}

\definecolor{boxlight}{HTML}{c7e0df}
\definecolor{boxdark}{HTML}{2C4443}

\newtcolorbox{interpretation}{
  colback=boxlight,
  colframe=boxdark,
  coltext=boxdark,
  boxsep=5pt,
  arc=2pt}
  
\newenvironment{infobox}
  {
  \begin{itemize}
  \renewcommand{\labelitemi}{
    \raisebox{-.7\height}[0pt][0pt]{
      {\setkeys{Gin}{width=3em,keepaspectratio}
        \includegraphics{img/key3.png}}
    }
  }
  \setlength{\fboxsep}{1em}
  \begin{interpretation}
  \item
  }
  {
  \end{interpretation}
  \end{itemize}
  }
\usepackage[]{natbib}
\bibliographystyle{plainnat}

\title{The STRAF Book}
\author{Alexandre Gouy and Martin Zieger}
\date{2021-09-26}

\begin{document}
\maketitle

{
\setcounter{tocdepth}{1}
\tableofcontents
}
\hypertarget{preface}{%
\chapter*{Preface}\label{preface}}
\addcontentsline{toc}{chapter}{Preface}

\hypertarget{what-is-this-book}{%
\section*{What is this book?}\label{what-is-this-book}}
\addcontentsline{toc}{section}{What is this book?}

This is the online version of \textbf{The STRAF Book}, which is currently under
active development. It is dedicated to the STRAF software, a web application
for the analysis of genetic data in forensic practice.

\hypertarget{forensic-and-population-genetics-lost-sisters}{%
\section*{Forensic and population genetics, lost sisters}\label{forensic-and-population-genetics-lost-sisters}}
\addcontentsline{toc}{section}{Forensic and population genetics, lost sisters}

Genetics has many faces, and forensic and population genetics are two of them.
If we were to summarise their respective scopes, we could say that the former
is the application of genetics to legal matters, and the latter aims at
understanding genetic differences within and between populations, a
fundamental matter in evolutionary biology.

Forensic genetics and population genetics have always been tightly linked
disciplines. This is likely because quite a number of questions they address
are similar. Even though problems in forensics and population genetics seem
different, they often correspond to the same question, simply phrased differently.

For example, DNA profiling, used in criminal investigations or parental testing,
aims at matching different DNA samples and understanding how related they are in
terms of DNA. In population genetics, a common goal is to characterise the
genetic diversity of a set of populations, by looking at how related individuals
are within and between populations. Both fields aim at \textbf{understanding} and
\textbf{quantifying} the \textbf{relatedness} of individuals based on their DNA.

Software and metrics developed in the population genetics for the study of the
evolution of species are now used routinely in forensic genetics practice.
But forensics is not just \emph{applied population genetics}. The legal implications
and unique situations encountered in the forensics world also led to the
development of relevant statistical tools and metrics with a more specific purpose.

\hypertarget{and-then-there-was-straf}{%
\section*{And then there was STRAF}\label{and-then-there-was-straf}}
\addcontentsline{toc}{section}{And then there was STRAF}

\textbf{STRAF} was born from the encounter of two scientists: a forensic geneticist and a
population geneticist, in 2017, in the beautiful city of Bern (Switzerland).
Martin came to visit a population genetics lab, where Alexandre was pursuing
his Ph.D.~thesis at that time. This encounter led to a fruitful collaboration
when they realised that some tools used in population genetics could be
leveraged by the forensics community.

This encounter led to a fruitful collaboration when they realised that some tools
used in population genetics could be leveraged by the forensics community. The
most striking example is the computation of forensics parameters, that describe
for example how good are our loci at discriminating samples. These
parameters were typically computed using a spreadsheet that had been created by
one of the suppliers of assays used to genotype samples. It is the mythical
PowerStats v1.2 spreadsheet, allowing to compute forensic statistics and allele
frequencies in Microsoft Excel. It has been since then removed from the Internet,
and forensic geneticists started sharing this spreadsheet among each other, circulating
almost secretly, ``under the cloak'' as French speakers would say.

As similar operations were done in routine in population genetics, we already had
some scripts for the analysis of STR data. Then, after we applied them to an existing
dataset, we decided to put everything into a web application so that the forensics
community could benefit from it.

A few weeks later, STRAF was born, and after four year, STRAF had become a
widely used tool by the forensics community, but not only.
It has been used as a support for teaching population genetics, and has
been used in evolutionary biology studies.
The positive reception of the software in the community motivated its
development over the years until the release of STRAF 2.0 in 2021.

\hypertarget{what-will-you-learn}{%
\section*{What will you learn?}\label{what-will-you-learn}}
\addcontentsline{toc}{section}{What will you learn?}

By reading this book, our hope is that you will:

\begin{itemize}
\item
  Get an overview of common \textbf{concepts} in forensic and population genetics
\item
  Learn how to use the \textbf{STRAF software} for STR data analysis through \textbf{practical applications}
\item
  Be able to \textbf{interpret} common metrics and analyses used in forensics practice
\end{itemize}

\hypertarget{outline}{%
\section*{Outline}\label{outline}}
\addcontentsline{toc}{section}{Outline}

The book is organised as follow:

\begin{itemize}
\item
  We'll start by an \textbf{Introduction} to essential forensic and population genetics concepts.
\item
  In \textbf{Chapter 1}, we will focus on data, from its generation to its preparation for
  downstream analysis in STRAF.
\item
  In \textbf{Chapter 2}, we will review \textbf{forensic parameters} that can be computed in STRAF,
  and discuss their interpretation.
\item
  In \textbf{Chapter 3}, we will review essential population genetics concepts and
  describe \textbf{population genetics indices} that can be computed in STRAF.
\item
  In \textbf{Chapter 4}, we will focus on \textbf{multivariate statistics} and how they can provide
  insights into population structure, with a particular focus on Principal Component
  Analysis (PCA) and Multidimensional Scaling (MDS), two widely used approaches in genetics.
\item
  In \textbf{Chapter 5}, we gather recommendations around potential next analysis steps
  by presenting STRAF's \textbf{file conversion} capabilities and useful methods implemented in
  \textbf{other software}.
\end{itemize}

\hypertarget{introduction}{%
\chapter*{Introduction}\label{introduction}}
\addcontentsline{toc}{chapter}{Introduction}

In this chapter, we will briefly introduce some essential concepts in genetics.

\hypertarget{dna-and-genetic-variation}{%
\section*{DNA and genetic variation}\label{dna-and-genetic-variation}}
\addcontentsline{toc}{section}{DNA and genetic variation}

Each of our cells contains 23 pairs of \textbf{chromosomes}, composed of a long \textbf{DNA}
(deoxyribonucleic acid) molecule. Under this somewhat barbaric name
is hiding a simple concept. This molecule is the support of the information
used by the body to function and development. This information is encoded by a
chain of \textbf{nucleotides} of four types that can be referred to using the letters
A (Adenine), T (Thymine), C (Cytosine) and G (Guanine).

Developments in biotechnologies enabled the ability to characterise the DNA of
individuals. These technologies also led to the discovery that this DNA varies
between individuals. This \textbf{genetic variation}, also called \textbf{polymorphism},
can be used to characterise individuals and populations based on their DNA.

\hypertarget{markers-of-polymorphism}{%
\section*{Markers of polymorphism}\label{markers-of-polymorphism}}
\addcontentsline{toc}{section}{Markers of polymorphism}

DNA variation can take different forms: it can for example be a
\textbf{Single Nucleotide Polymorphism} (\textbf{SNP}), when a mutation occured and changed a
nucleotide at a given position in the genome. In that case, we would observe
different nucleotides in a population at a single position.

There are also \textbf{insertions} and \textbf{deletions} (sometimes referred to as
\textbf{InDels}), of one or multiple nucleotides.

Finally, other markers of genetic variation are
\textbf{Copy Number Variants} (\textbf{CNVs}), when a sequence is repeated a certain number
of times. They can be and contain more or less repetitive units.

\textbf{Short Tandem Repeats} (\textbf{STRs}) a type of genetic polymorphism consisting
in short sequences from 2 to 7 base pairs that are repeated a certain number of time.
As the number of repeats varies among individuals, characterizing their
length can be useful to identify individuals.

\hypertarget{polymorphism-and-forensics}{%
\section*{Polymorphism and forensics}\label{polymorphism-and-forensics}}
\addcontentsline{toc}{section}{Polymorphism and forensics}

\hypertarget{dna-profiling-and-typing}{%
\subsection*{DNA profiling and typing}\label{dna-profiling-and-typing}}
\addcontentsline{toc}{subsection}{DNA profiling and typing}

As DNA varies between individuals, DNA typing became a central element of the forensic
scientist toolkit. For example, typical questions forensic genetics aims at
answering include:

\begin{itemize}
\item
  What is the probability that a randomly-picked person in a population would
  match the individual of interest in terms of DNA?
\item
  Which proportion of the population has the same combination of
  genetic variants as the sample of interest?
\end{itemize}

\hypertarget{the-role-of-population-genetics}{%
\subsection*{The role of population genetics}\label{the-role-of-population-genetics}}
\addcontentsline{toc}{subsection}{The role of population genetics}

To answer these questions, it is crucial to first get a good characterisation
of \textbf{genetic variants} (or \textbf{alleles}) frequencies in populations of interest,
at different \textbf{loci} across the genome. Indeed, these frequencies can vary
widely among populations.

\textbf{Digression - Why STRs and not other markers?}

Comparing to whole-genome sequencing. They have a high \textbf{power of discrimination}
because they have a much higher diversity (their mutation rate is higher than
other common markers).

\hypertarget{straf-and-population-data-analysis}{%
\section*{STRAF and population data analysis}\label{straf-and-population-data-analysis}}
\addcontentsline{toc}{section}{STRAF and population data analysis}

In this context, STRAF has been designed to facilitate the analysis of
\textbf{population data} in forensic genetics.

\hypertarget{importing-data}{%
\chapter{Importing data}\label{importing-data}}

Work in progress.

\hypertarget{str-data}{%
\section{STR data}\label{str-data}}

\begin{itemize}
\tightlist
\item
  Observed values: genotypes for each individual, at each locus.
\item
  Potentially two values observed per individual and per locus, if diploid markers.
\item
  Value = can be anything but tipically correspond, for STR markers, to the length.
\item
  Point alleles
\end{itemize}

\hypertarget{input-data-format}{%
\section{Input data format}\label{input-data-format}}

STRAF's input file is a text file containing the genotypes of each sample:

\begin{itemize}
\tightlist
\item
  The first column, named \textbf{ind}, needs to contain the sample ID
\item
  The second column, , named \textbf{pop}, contains the population ID (this column must exist even if a single population is studied)
\item
  The next columns correspond to genotypes: for haploid samples, one column per locus must be reported; for diploid data, two columns per locus (with the same name)
\item
  Genotypes must be encoded as numbers (STRAF accepts point alleles)
\item
  Missing data (e.g.~null alleles) must be indicated with a ``0''.
\end{itemize}

For diploid data, the table should look like this:

\begin{longtable}[]{@{}llllll@{}}
\toprule
ind & pop & Locus1 & Locus1 & Locus2 & Locus2\tabularnewline
\midrule
\endhead
A & Bern & 12 & 14 & 17 & 17\tabularnewline
B & Bern & 14 & 14 & 13 & 15.2\tabularnewline
C & Lausanne & 12 & 16 & 15.2 & 17\tabularnewline
\bottomrule
\end{longtable}

For haploid data, the table would look like this:

\begin{longtable}[]{@{}llll@{}}
\toprule
ind & pop & Locus1 & Locus2\tabularnewline
\midrule
\endhead
A & Bern & 12 & 17\tabularnewline
B & Bern & 14 & 13\tabularnewline
C & Lausanne & 12 & 15.2\tabularnewline
\bottomrule
\end{longtable}

\hypertarget{generating-the-input-data-from-excel}{%
\section{Generating the input data from Excel}\label{generating-the-input-data-from-excel}}

It only takes a few steps to generate an input file in a format that is suitable
for use in STRAF. From Excel, for example, we can start from a spreadsheet looking
like this:

Then, one simply needs to save this table as a tab-delimited text file. This can be
achieved by clicking on \texttt{Save\ As} \textgreater{} \texttt{Text\ (Tab-delimited)\ (*.txt)}

\hypertarget{uploading-the-data-to-straf}{%
\section{Uploading the data to STRAF}\label{uploading-the-data-to-straf}}

Coming soon.

\hypertarget{common-issues}{%
\section{Common issues}\label{common-issues}}

Even though you've been very careful in the generation of STRAF's input file,
it is possible that you still run into an error after uploading the file to STRAF.
In case STRAF cannot read your input file, we've put together a checklist to identify
common issues with the input file.

\textbf{Input file checklist}

\begin{itemize}
\tightlist
\item
  Check input parameters in the sidebar: do they actually correspond to the input data?
\item
  Check locus names: are they all different for haploid data? Do both columns for a single locus for diploid data have the exact same name?
\item
  Check that all missing data have been encoded with a ``0''
\item
  Try to remove any special characters from sample and locus names
\item
  Check for the presence of empty spaces at the end of each line
\item
  Check if alleles are exclusively encoded with numbers
\item
  Check if values are separated by tabs and not spaces
\item
  Check if the first two columns are names ``ind'' and ``pop''
\end{itemize}

\hypertarget{having-a-first-look-at-the-data}{%
\section{Having a first look at the data}\label{having-a-first-look-at-the-data}}

\hypertarget{forensic-parameters}{%
\chapter{Forensic parameters}\label{forensic-parameters}}

In this chapter, we'll introduce a few equations. Do not be afraid! The goal of
this chapter is to translate each of them into plain English.

\hypertarget{how-to-compute-forensic-parameters-in-straf}{%
\section{How to compute forensic parameters in STRAF}\label{how-to-compute-forensic-parameters-in-straf}}

\hypertarget{random-match-probability-pm}{%
\section{Random match probability (PM)}\label{random-match-probability-pm}}

The \textbf{Random match probability}, or probability of matching (PM), is defined as
the probability of observing a random match between two individuals.

\textbf{Formula}

\[
PM = \sum_i (G_i)^2,
\]
where \(G_i\) is the frequency of the genotype \(i\) at a given locus in the population.

\textbf{Interpretation}

Computing \(PM\) means calculating, for a given locus, the frequency of each
genotypes. Then we take the square of each frequency, i.e.~we multiply it by itself.
Finally, we sum the values of each genotype.

The intuition behind it is that if we have a random match in a population by looking
at a singly locu, it means that our two samples have the same genotype at that locus.
In terms of probability, this will happen as many times as we .
The probability of sampling a specific genotype in the population is equal to its frequency.

Say the genotype ``12-14'' has a frequency of 5\% in the population, the probability of
having a random match between two individuals having the same genotype is 0.05 x 0.05.
To get an overall probability of matching, we sum this over all possible genotypes
in our population.

\hypertarget{power-of-discrimination-pd}{%
\section{Power of Discrimination (PD)}\label{power-of-discrimination-pd}}

The power of discrimination (PD) is defined as the probability of
discriminating between two unrelated individuals.

\textbf{Formula}

\[
PD = 1 - PM
\]

\textbf{Interpretation}

PD is simply 1 - PM. Instead of looking at the probability of matching, we are
interested in the probability of ``not matching'', i.e.~the probability of discrimination.

\hypertarget{gene-diversity}{%
\section{Gene diversity}\label{gene-diversity}}

Genetic diversity (\(GD\)), or expected heterozygosity (\(H_{\mathrm{exp}}\)), is
computed using the following estimator:

\textbf{Formula}

\[
  H_{\mathrm{exp}} = GD = \frac{n}{n - 1} \left( 1 - \sum_{i=1}^{n}(p_i)^2 \right),
\]

where \(n\) is the number of gene copies sampled and \(p_i\) is the
frequency of the \(i^{th}\) allele in the population.

\textbf{Interpretation}

Coming soon.

\hypertarget{polymorphism-information-content-pic}{%
\section{Polymorphism Information Content (PIC)}\label{polymorphism-information-content-pic}}

\textbf{Formula}

Polymorphism Information Content (PIC) is computed as follow:

\[
PIC = 1 - \sum_{i=1}^{n} p_i^2 - \sum_{i=1}^{n-1} \sum_{j=i+1}^{n} 2p_i^2p_j^2,
\]
where \(p_i\) and \(p_j\) are allele frequencies.

\textbf{Interpretation}

The PIC can be interpreted as:

\begin{itemize}
\item
  the probability that the maternal and paternal alleles of a child are
  deducible
\item
  or, the probability of being able to deduce which allele a
  parent has transmitted to the child.
\end{itemize}

\hypertarget{power-of-exclusion-pe}{%
\section{Power of Exclusion (PE)}\label{power-of-exclusion-pe}}

\textbf{Formula}

The power of exclusion (\(PE\)) is defined as:

\[
PE = h^2\left(1 - 2hH^2\right),
\]
where \(h\) is the proportion of heterozygous individuals and \(H\) the
proportion of homozygous individuals in the population sample.

\textbf{Interpretation}

Coming soon.

\hypertarget{typical-paternity-index-tpi}{%
\section{Typical Paternity Index (TPI)}\label{typical-paternity-index-tpi}}

Finally, the typical paternity index (\(TPI\)) reflects the ``mean PI for
random non-excluded men`` for a given locus.

\textbf{Formula}

Let \(H\) be the frequency of homozygotes, then

\[
TPI = \frac{1}{2H}
\]

\textbf{Interpretation}

Coming soon.

\hypertarget{population-genetics-indices}{%
\chapter{Population genetics indices}\label{population-genetics-indices}}

\hypertarget{population-genetics-concepts}{%
\section{Population genetics concepts}\label{population-genetics-concepts}}

\textbf{Hardy-Weinberg equilibrium}

A population is considered at Hardy-Weinberg equilibrium when

Why is it important to check? If a locus presents a significant deviation from
HWE, it means that a process is influencing the distribution of allele and genotype
frequencies in the population.

\begin{itemize}
\item
  inbreeding
\item
  population structure
\item
  locus is under selection. This is very unlikely that STR loci as they are
  supposed to evolve neutrally. However, they could be found near loci under selection
\end{itemize}

In forensics, we need to assume HWE as indices computed (for example, a match probability)
would be biased if

\begin{itemize}
\tightlist
\item
  Population structure
\end{itemize}

Individuals closer from each other are in general more likely to mate with each other.

\hypertarget{how-to-compute-population-genetics-parameters-in-straf}{%
\section{How to compute population genetics parameters in STRAF}\label{how-to-compute-population-genetics-parameters-in-straf}}

\hypertarget{population-genetics-parameters}{%
\section{Population genetics parameters}\label{population-genetics-parameters}}

\begin{itemize}
\item
  Heterozygosities
\item
  F-statistics: FIS and FST
\end{itemize}

\textbf{One concept, multiple estimators.}

Several \textbf{estimators} of Fst exist (for example, Weir and Cockerham's, Nei's,
Hudson's FST). It's like if each population geneticist decided to develop their
own estimator! Why is that? In statistics, what we call an \textbf{estimator}
is. It is important to keep in mind that these estimators rely on a specific \textbf{model},
with underlying assumptions. It explains why some estimators are more or less reliable
depending on the case and observed data, and each of them has been developed for
a different situation.

\hypertarget{what-is-linkage-disequilibrium}{%
\section{What is linkage disequilibrium?}\label{what-is-linkage-disequilibrium}}

\textbf{Coming soon.}

\hypertarget{how-to-compute-ld-in-straf}{%
\section{How to compute LD in STRAF}\label{how-to-compute-ld-in-straf}}

\hypertarget{multivariate-statistics}{%
\chapter{Multivariate statistics}\label{multivariate-statistics}}

In this chapter, we will describe the two multivariate statistics methods implemented
in STRAF, \textbf{Principal Component Analysis} (\textbf{PCA}) and \textbf{Multidimensional Scaling} (\textbf{MDS}).

\hypertarget{principal-component-analysis-pca}{%
\section{Principal Component Analysis (PCA)}\label{principal-component-analysis-pca}}

PCA is a method of \textbf{dimensionality reduction}. What is does it that it tries to
\textbf{capture most of the variation} present in the data and \textbf{project} it onto a
small number of new axes called \textbf{principal components} (PCs).

This is a useful method to capture variation from a large number of variables and
allows to discover hidden patterns by increasing interpretability.

In our case, if we consider that \textbf{each allele at each locus} is a variable, and that
our individual observations are the presence / absence of each allele for each sample,
we end up with a highly dimensional dataset (we have as many variables as we have
alleles!). It gets even worse if you analyse genome sequences, where you can have millions
of variables in your dataset. This is definitely not an interpretable dataset if you
are not able to easily extract relevant information.

PCA allows to bring most of the variation existing between our samples onto a few
axes capturing most of the variation (PCs).

In STRAF, you can perform a PCA by going into the \textbf{PCA - MDS} tab and checking
the following \emph{Run and plot a PCA} checkbox.

It will trigger the PCA computation and return a graph. You are also able to
download to coordinates on different PCs (also called eigenvectors).

\begin{interpretation}
\textbf{Interpreting PCA results}

PCA has become a popular tool in population genetics, as the relationships between
individuals on the PCA projection tends to reflect their genetic \textbf{relatedness}. The
closer individuals are on the PCA projection, the more genetically related they tend to be.

It can also be used as a \textbf{quality control} tool. For example, if a sample is very far
from all the others, including the ones that are part of the same population,
it is possible that there is an issue with the data. One would need to inspect
the raw data and check if any strange pattern can be found.

It is important to be aware of the influence of non demographic processes on
the PCA projection. For example, \textbf{imbalanced sample sizes} between populations
can drive some patterns. When populations are sampled unevenly, the projection will
be \textbf{distorted} and distances observed on the projection can be driven by these
differences and not by their evolutionary history.

Hence, as multiple processes influence the results, PCA should remain an
\textbf{exploratory approach} and further analyses should be conducted before drawing
any major conclusions on the relationships between populations and individuals.

\end{interpretation}

\hypertarget{multidimensional-scaling-mds}{%
\section{Multidimensional Scaling (MDS)}\label{multidimensional-scaling-mds}}

An \textbf{MDS} is conceptually similar to a PCA. One of the main differences is
that it takes as input data in a different format. As PCA uses raw genotypes and
can accommodate data at the individual level, \textbf{pairwise distances} between data
points are used for an MDS.

In forensics practice, it is typically used to compare \textbf{populations} and not
\textbf{individuals}, even though it would be theoretically possible.

As \textbf{pairwise distances between populations} are used, and MDS can be based on
any genetic distance: one could for example compute pairwise \textbf{FSTs},
\textbf{Nei's genetic distance}, or any other metric.

Based on these distances, the MDS will \textbf{project} the populations onto a
\textbf{lower-dimensional space}. This projection facilitates the interpretation of
relationships between populations.

\begin{interpretation}
\textbf{Interpreting MDS results}

Just like with a PCA, populations the \textbf{closer} populations are on the MDS projection,
the more \textbf{genetically related} the tend to be, based on the markers that have been
used to compute their pairwise genetic distances.

Again, like for the PCA, the MDS should remain an
\textbf{exploratory approach} and further analyses should be conducted before drawing
any major conclusions on the relationships between populations and individuals.

\end{interpretation}

\hypertarget{file-conversion}{%
\chapter{File conversion}\label{file-conversion}}

As STRAF is a web application and can be used simultaneously by multiple users,
computing resources are limited. Therefore, the most computationally intensive analyses
are not available in STRAF. In order to ease the path to other software,
file conversion utilities have been implemented. It is possible to convert
the input file to the \textbf{Genepop}, \textbf{Arlequin} and \textbf{Familias} formats. They are
all available in the \textbf{File conversion} tab of the application.

\hypertarget{how-to-convert-a-file-in-straf}{%
\section{How to convert a file in STRAF}\label{how-to-convert-a-file-in-straf}}

Once you have imported your file, it is straightforward to conevrt it to another
format. You can go to the \textbf{File conversion} tab and click on one of the following
buttons to download your genotypes in another format.

\hypertarget{genepop-and-arlequin-formats}{%
\section{Genepop and Arlequin formats}\label{genepop-and-arlequin-formats}}

\textbf{Genepop} and \textbf{Arlequin} softwares implement several population genetics methods,
including ones that are part of standard forensics practice:
* linkage disequilibrium computation
* Hardy-Weinberg tests

STRAF currently implements these computations, however the ones implemented in
Genepop are overall more reliable as they can rely on more permutations. They
are overall preferable to the HW and LD tests implemented in STRAF.

\hypertarget{familias}{%
\section{Familias}\label{familias}}

Here a file containing allele frequencies is created. This file can be used in
Familias to provide allele frequencies reference.

\end{document}
